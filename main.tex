\documentclass{article}
\usepackage[utf8]{inputenc}
\usepackage{outlines}
\usepackage{hyperref}
\usepackage{amsmath}
\hypersetup{
    colorlinks=true,
    linkcolor=cyan,
    filecolor=blue,      
    urlcolor=blue,
}
\urlstyle{same}

\usepackage{enumitem}
\setenumerate[1]{label=\Roman*.}
\setenumerate[2]{label=\Alph*.}
\setenumerate[3]{label=\roman*.}
\setenumerate[4]{label=\alph*.}

\title{3 Buddy Problem}

\begin{document}

\maketitle

\section{The Problem}
Everyone needs a coding buddy. So we have a column associated with each user called “Buddy”. But now we need a group to have 3 people. How do we represent this group of three people?

\section{Some Solutions}
\begin{outline}[enumerate]
    \1 Add a column for the third person
        \2 What happens if we need to put a fourth person? fifth?
        \2 Do we really want so many empty cells associated with each of the other pairs?
            \3 Let’s try and avoid nulls
    \1 Comma separate the buddies columns
        \2 Not atomic
            \3 \href{https://www.1keydata.com/database-normalization/first-normal-form-1nf.php#:~:text=An\%20atomic\%20value\%20is\%20a,columns\%20that\%20are\%20closely\%20related.}{1keydata.com}
            \3 \href{https://dba.stackexchange.com/questions/2342/what-is-atomic-relation-in-first-normal-form}{dba.stackexchange}
    \1 Create a new row for every pair within the new group
        \2 This is going to get really ugly really quickly. Imagine we need 1 group with 4 people. 8 rows?
        \2 $\dfrac{n(n-1)}{2}$ pairs for any group of size n.
        \2 $O(n^2)$ space for this task should be viewed as unacceptable.
    \1 Create a new column called “Team\_Name”
        \2 Create a new table, “Teams'' with appropriate attributes and a primary key
        \2 There will be a one to many relationship between the CLASS\_ROASTER table and the TEAMS table via a foreign key relationship.
        \2 We should note that the team does not need to know who belongs to it
    \1 Change the column data type to an Array data type (Prof. Zombory)
        \2 \href{https://www.postgresql.org/docs/9.1/arrays.html}{Arrays Documentation}
        \2 This is probably the simplest answer to maintain normal form(s) within the schema.
        \2 This is also a scalable solution, since if a team requires more teammates, it is very simple to add the new teammate.
        \2 However, Arrays are not a datatype supported by all RDBMS.
            \3 MSSQL does have \href{https://docs.microsoft.com/en-us/sql/relational-databases/xml/xml-data-type-and-columns-sql-server?view=sql-server-ver15}{XML} and it also handles JSON. Especially in the age of big data some of the relational concepts have been challenged or altogether left behind.     
        \2 One issue that does remain with this solution is how would we store other attributes about the team, such as it's mascot or primary color.
\end{outline}

<<<<<<< HEAD
\section{My Opinion}  I believe solution IV to be the correct solution. While both solutions IV and V allow us to maintain a scalable, robust, minimal to no redundancies, and reusable design, solution IV seperates concerns more efficiently, and allows for more attributes about all teams to be added and modified more easily. It is now very easy and hopefully obvious how to add information about the teams going forward.
=======
\section{My Opionon}  I believe solution IV to be the correct solution. It will allow us to maintain a scalable, robust, minimal to no redundancies, and reusable design. It is now very easy and hopefully obvious how to add information about the teams going forward.
>>>>>>> parent of 6fd758a... Fixed a typo





  




\end{document}
